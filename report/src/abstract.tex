% !TeX root = ../main.tex

\begin{abstract}
  参数化设计是一种基于规则和算法的设计方法, 它可以实现复杂形态的生成和控制, 并提高设计的效率和质量.
  本文以 Rhino 和 Grasshopper 为平台, 探讨了参数化设计中的几个关键问题: 曲线打断, 容差合并, 拓扑优化.
  首先, 介绍了参数化设计的概念, 特点和发展历程, 以及 Rhino 和 Grasshopper 软件的功能和特点.
  其次, 分析了基于树形数据的曲线打断方法, 并给出了相应的 Grasshopper 实现.
  然后, 讨论了容差合并的重要性和问题, 并提出了三种容差合并算法: 朴素容差合并算法, 基于近邻搜索的容差合并算法, 基于 k-d Tree 的容差合并算法.
  对比了这三种算法的原理, 复杂度和效率, 并给出了相应的 C\# 和 Python 源码.
  接着, 分析了拓扑优化的重要性, 并介绍了拓扑优化的方法.
  最后, 通过一个实际案例, 展示了拓扑优化的实际应用, 并给出了优化流程, 优化结果和优化意义.

  \paragraph{关键词:}
  参数化设计; Rhino; Grasshopper; 曲线打断; 容差合并; 拓扑优化
\end{abstract}

\renewcommand{\abstractname}{Abstract}

\begin{abstract}
  Parametric design is a design method based on rules and algorithms, which can achieve the generation and control of complex forms, and improve the efficiency and quality of design.
  This paper explores several key issues in parametric design using Rhino and Grasshopper as platforms: curve-breaking, tolerance merging, topological simplification, and repair.
  Firstly, the concept, characteristics, and development history of parametric design are introduced, as well as the functions and features of Rhino and Grasshopper software.
  Secondly, the curve-breaking method based on tree-structured data is analyzed, and the corresponding Grasshopper implementation is provided.
  Then, the importance and problems of tolerance merging are discussed, and three tolerance merging algorithms are proposed: naive tolerance merging algorithm, tolerance merging algorithm based on nearest neighbor search, and tolerance merging algorithm based on k-d tree.
  The principles, complexity, and efficiency of these three algorithms are compared, and corresponding C\# and Python source code are provided.
  Furthermore, the importance of topological optimization is analyzed, and the methods for topological optimization are introduced.
  Finally, through a practical case, the practical application of topological optimization is demonstrated, and the optimization process, results, and significance are presented.

  \paragraph{Keywords:}
  Parametric Design; Rhino; Grasshopper; Curve Breaking; Tolerance Merging; Topological Optimization
\end{abstract}
