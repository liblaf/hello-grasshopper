% !TeX root = ../main.tex

% protected override void SolveInstance(IGH_DataAccess DA)
% {
%     DA.GetDataTree(name: "Curves", tree: out GH_Structure<GH_Curve> curves);
%     GH_Structure<GH_Curve> result = new GH_Structure<GH_Curve>();
%     foreach (GH_Path path in curves.Paths)
%     {
%         for (int i = 0; i < curves[path].Count; ++i)
%         {
%             Curve curve = curves[path][i].Value;
%             List<double> parameters = new List<double>();
%             foreach (GH_Curve gh_cutter in curves.AllData(skipNulls: true))
%             {
%                 Curve cutter = gh_cutter.Value;
%                 if (cutter == curve)
%                     continue;
%                 CurveIntersections intersections = Intersection.CurveCurve(
%                     curve,
%                     cutter,
%                     tolerance: 0.0,
%                     overlapTolerance: 0.0
%                 );
%                 parameters.AddRange(
%                     intersections.Select(intersection => intersection.ParameterA)
%                 );
%             }
%             parameters.Sort();
%             parameters = parameters
%                 .Prepend(curve.Domain.Min)
%                 .Append(curve.Domain.Max)
%                 .ToList();
%             List<GH_Curve> segments = new List<GH_Curve>();
%             for (int j = 0; j + 1 < parameters.Count; ++j)
%             {
%                 double t0 = parameters[j];
%                 double t1 = parameters[j + 1];
%                 segments.Add(new GH_Curve(curve.Trim(t0, t1)));
%             }
%             result.AppendRange(data: segments, path: path.AppendElement(i));
%         }
%     }
%     DA.SetDataTree(paramIndex: 0, tree: result);
% }

% Grasshopper 中的曲线打断工具可以实现曲线的分割, 但存在一些问题:
% Grasshopper 中的曲线打断工具存在一些问题:
% - 无法处理树形结构存储的曲线
% - 丢失曲线的属性信息
% 因此, 我们实现了一个新的曲线打断插件, 核心代码如上.

% 假设你是一名资深的参数化软件研究员, 你正在撰写一篇关于 Rhino 和 Grasshopper 的实际应用的论文. 现在, 请根据上述代码和信息, 撰写曲线打断一小节.

\section{曲线打断}

\subsection{曲线打断简介}

曲线打断是参数化设计中常用的一种技术, 可以通过在曲线上添加断点将其分割成多段.
Grasshopper 是 Rhino 软件中一款强大的可视化编程工具, 可用于实现参数化设计中的曲线打断操作.
本节将介绍曲线打断的原理及其在参数化设计中的应用.

在参数化设计中, 曲线打断是一个常见的操作, 可以让设计师对曲线进行细粒度的控制.
通过曲线打断, 设计师可以将曲线划分成多个片段, 并对其进行个别处理.
这种操作可以用于构建复杂的几何形状, 进行曲线上特定区域的编辑, 以及生成符合设计要求的曲线样式.

通过 Grasshopper 的强大功能, 曲线打断可以快速实现.
设计师可以使用 Grasshopper 的曲线工具创建自定义的曲线, 然后使用打断工具将其分割成多个片段.
打断工具允许设计师在曲线上添加断点, 并根据需要调整断点的位置和数量.
设计师还可以使用Grasshopper的其他组件对打断后的曲线片段进行进一步的操作和修改, 以满足设计的要求.

曲线打断的应用广泛存在于各个设计领域.
在建筑设计中, 曲线打断可以用于生成具有动态感的曲线体量, 提升建筑的空间层次感.
在产品设计中, 曲线打断可以用于创建独特的产品造型, 增强产品的美观性和功能性.
在景观设计中, 曲线打断可以用于创造流畅的路径线路, 使景观布局更加灵活多样.

\subsection{基于树形数据的曲线打断方法}

曲线打断方法的有效实现对于参数化软件的应用具有重要意义.
尽管在 Grasshopper 中提供了曲线打断工具, 但该工具存在一些局限性, 例如无法处理树形结构存储的曲线以及丢失曲线的属性信息.
为了解决这些问题, 我们在本研究中提出了一个新的曲线打断插件.

\inputcode[
  label          = lst:split-curves-core,
  minted options = {
      firstline = 45,
      lastline  = 86,
    },
][SplitCurves.cs]{csharp}{../csharp/HelloGrasshopper/SplitCurves.cs}

我们的曲线打断插件的核心代码如 \cref{lst:split-curves-core} 所示, 完整代码参见 \cref{lst:split-curves}.
首先, 我们通过输入参数 \mintinline{csharp}{Curves} 获取一个曲线树结构, 该结构以路径为单位存储了一组曲线.
然后, 我们遍历树结构中的所有路径, 并逐个处理其中的曲线.

对于每个曲线, 我们首先创建一个空的曲线结果集合, 用于存储被分割后的曲线段.
然后, 我们针对当前曲线和其他曲线进行相交检测.
如果两条曲线相同, 我们将继续处理下一条曲线.
否则, 我们使用 \mintinline{csharp}{Intersection.CurveCurve} 方法计算曲线相交的交点, 并将其中一个曲线的参数值加入到参数集合中.

接下来, 我们对参数集合进行排序, 并将当前曲线的起始和结束参数值加入到参数集合中.
然后, 我们以参数集合中相邻参数值的位置为界, 将当前曲线分割成多个曲线段, 并将这些曲线段作为新的曲线对象加入到曲线结果集合中.

最后, 我们将曲线结果集合添加到树结构中的对应路径中, 完成所有曲线的打断操作.
最终, 通过输出参数可以获取到完整的打断后的曲线树结构.

我们的曲线打断插件能够有效处理树形结构存储的曲线, 并且能够保留曲线的属性信息.
这为参数化软件的用户提供了更灵活和便捷的曲线操作方式, 进一步提升了其应用领域的实际价值.

本研究的实现为曲线打断方法在 Rhino 和 Grasshopper 中的应用提供了一种可靠而高效的解决方案.
通过借鉴和优化现有工具的不足之处, 我们为参数化软件的开发和使用者提供了一种优化的曲线打断功能, 从而进一步推进了参数化设计的应用和发展.
