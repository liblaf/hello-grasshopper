% !TeX root = ../main.tex

\section{结论}

本文以 Rhino 和 Grasshopper 为平台, 探讨了参数化设计中的曲线打断, 容差合并, 拓扑优化等问题.
主要内容和结论如下:
\begin{itemize}
  \item 本文提出了一种基于树形数据的曲线打断方法, 可以有效地将复杂的曲线网络分割成单独的曲线段, 并保持其拓扑关系.
        该方法可以应用于参数化设计中的多种场景, 如网格生成, 曲面分割, 结构优化等.
  \item 本文分析了容差合并的重要性和问题, 并比较了三种容差合并算法: 朴素容差合并算法, 基于近邻搜索的容差合并算法和基于 k-d Tree 的容差合并算法.
        结果表明, 基于 k-d Tree 的容差合并算法具有最高的效率, 可以在较短的时间内处理大量的点集数据.
        本文还提出了一种点集的排序方法, 可以进一步提高容差合并的稳定性.
  \item 本文介绍了拓扑优化的重要性和方法, 并以一个实际案例为例, 展示了如何利用 Grasshopper 和 C\# 插件实现拓扑优化的功能.
        通过拓扑优化, 可以消除参数化设计中产生的多余或错误的元素, 提高模型的质量和可用性.
\end{itemize}

本文为参数化设计提供了一些有用的工具和技巧, 希望能够为参数化设计的发展和应用带来一些启示和帮助.
当然, 参数化设计还有许多未解决的问题和挑战, 需要进一步的研究和探索.
本文仅是一个初步的尝试.
